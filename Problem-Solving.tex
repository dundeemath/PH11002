% Options for packages loaded elsewhere
% Options for packages loaded elsewhere
\PassOptionsToPackage{unicode}{hyperref}
\PassOptionsToPackage{hyphens}{url}
\PassOptionsToPackage{dvipsnames,svgnames,x11names}{xcolor}
%
\documentclass[
  a4paper,
  DIV=11,
  numbers=noendperiod,
  oneside]{scrreprt}
\usepackage{xcolor}
\usepackage{amsmath,amssymb}
\setcounter{secnumdepth}{5}
\usepackage{iftex}
\ifPDFTeX
  \usepackage[T1]{fontenc}
  \usepackage[utf8]{inputenc}
  \usepackage{textcomp} % provide euro and other symbols
\else % if luatex or xetex
  \usepackage{unicode-math} % this also loads fontspec
  \defaultfontfeatures{Scale=MatchLowercase}
  \defaultfontfeatures[\rmfamily]{Ligatures=TeX,Scale=1}
\fi
\usepackage{lmodern}
\ifPDFTeX\else
  % xetex/luatex font selection
\fi
% Use upquote if available, for straight quotes in verbatim environments
\IfFileExists{upquote.sty}{\usepackage{upquote}}{}
\IfFileExists{microtype.sty}{% use microtype if available
  \usepackage[]{microtype}
  \UseMicrotypeSet[protrusion]{basicmath} % disable protrusion for tt fonts
}{}
\makeatletter
\@ifundefined{KOMAClassName}{% if non-KOMA class
  \IfFileExists{parskip.sty}{%
    \usepackage{parskip}
  }{% else
    \setlength{\parindent}{0pt}
    \setlength{\parskip}{6pt plus 2pt minus 1pt}}
}{% if KOMA class
  \KOMAoptions{parskip=half}}
\makeatother
% Make \paragraph and \subparagraph free-standing
\makeatletter
\ifx\paragraph\undefined\else
  \let\oldparagraph\paragraph
  \renewcommand{\paragraph}{
    \@ifstar
      \xxxParagraphStar
      \xxxParagraphNoStar
  }
  \newcommand{\xxxParagraphStar}[1]{\oldparagraph*{#1}\mbox{}}
  \newcommand{\xxxParagraphNoStar}[1]{\oldparagraph{#1}\mbox{}}
\fi
\ifx\subparagraph\undefined\else
  \let\oldsubparagraph\subparagraph
  \renewcommand{\subparagraph}{
    \@ifstar
      \xxxSubParagraphStar
      \xxxSubParagraphNoStar
  }
  \newcommand{\xxxSubParagraphStar}[1]{\oldsubparagraph*{#1}\mbox{}}
  \newcommand{\xxxSubParagraphNoStar}[1]{\oldsubparagraph{#1}\mbox{}}
\fi
\makeatother


\usepackage{longtable,booktabs,array}
\usepackage{calc} % for calculating minipage widths
% Correct order of tables after \paragraph or \subparagraph
\usepackage{etoolbox}
\makeatletter
\patchcmd\longtable{\par}{\if@noskipsec\mbox{}\fi\par}{}{}
\makeatother
% Allow footnotes in longtable head/foot
\IfFileExists{footnotehyper.sty}{\usepackage{footnotehyper}}{\usepackage{footnote}}
\makesavenoteenv{longtable}
\usepackage{graphicx}
\makeatletter
\newsavebox\pandoc@box
\newcommand*\pandocbounded[1]{% scales image to fit in text height/width
  \sbox\pandoc@box{#1}%
  \Gscale@div\@tempa{\textheight}{\dimexpr\ht\pandoc@box+\dp\pandoc@box\relax}%
  \Gscale@div\@tempb{\linewidth}{\wd\pandoc@box}%
  \ifdim\@tempb\p@<\@tempa\p@\let\@tempa\@tempb\fi% select the smaller of both
  \ifdim\@tempa\p@<\p@\scalebox{\@tempa}{\usebox\pandoc@box}%
  \else\usebox{\pandoc@box}%
  \fi%
}
% Set default figure placement to htbp
\def\fps@figure{htbp}
\makeatother


% definitions for citeproc citations
\NewDocumentCommand\citeproctext{}{}
\NewDocumentCommand\citeproc{mm}{%
  \begingroup\def\citeproctext{#2}\cite{#1}\endgroup}
\makeatletter
 % allow citations to break across lines
 \let\@cite@ofmt\@firstofone
 % avoid brackets around text for \cite:
 \def\@biblabel#1{}
 \def\@cite#1#2{{#1\if@tempswa , #2\fi}}
\makeatother
\newlength{\cslhangindent}
\setlength{\cslhangindent}{1.5em}
\newlength{\csllabelwidth}
\setlength{\csllabelwidth}{3em}
\newenvironment{CSLReferences}[2] % #1 hanging-indent, #2 entry-spacing
 {\begin{list}{}{%
  \setlength{\itemindent}{0pt}
  \setlength{\leftmargin}{0pt}
  \setlength{\parsep}{0pt}
  % turn on hanging indent if param 1 is 1
  \ifodd #1
   \setlength{\leftmargin}{\cslhangindent}
   \setlength{\itemindent}{-1\cslhangindent}
  \fi
  % set entry spacing
  \setlength{\itemsep}{#2\baselineskip}}}
 {\end{list}}
\usepackage{calc}
\newcommand{\CSLBlock}[1]{\hfill\break\parbox[t]{\linewidth}{\strut\ignorespaces#1\strut}}
\newcommand{\CSLLeftMargin}[1]{\parbox[t]{\csllabelwidth}{\strut#1\strut}}
\newcommand{\CSLRightInline}[1]{\parbox[t]{\linewidth - \csllabelwidth}{\strut#1\strut}}
\newcommand{\CSLIndent}[1]{\hspace{\cslhangindent}#1}



\setlength{\emergencystretch}{3em} % prevent overfull lines

\providecommand{\tightlist}{%
  \setlength{\itemsep}{0pt}\setlength{\parskip}{0pt}}



 


\usepackage{unicode-math}%
\setmathfont{XITS Math}%
\usepackage{fontspec}%
\setmainfont[Ligatures ={Common, TeX}, Scale=1, RawFeature={+cpsp}]{XITS} 
%Numbers={Lining,Proportional},Ligatures ={Common, TeX},RawFeature={+tnum,+cpsp,+frac},
\setsansfont[RawFeature={+cpsp},Scale=MatchLowercase]{Helvetica Neue}%
\setmonofont[Scale=0.78]{MesloLGS NF}%

\usepackage[a4paper,%
margin=2.5cm,%
bottom=3cm,%
top=3cm]{geometry}%
\usepackage{afterpage}% for "\afterpage"
\usepackage{xcolor}%
\definecolor{dundeeblue}{HTML}{4365E2}%
\usepackage{pagecolor}% With option pagecolor={somecolor or none}
%% For nice tables
\usepackage{booktabs}%
\usepackage{longtable}%
\usepackage{array}%
\usepackage{multirow}%
\usepackage{wrapfig}%
\usepackage{float}%
\usepackage{colortbl}%
\usepackage{pdflscape}%
\usepackage{tabu}%
%\usepackage{threeparttable}%
%\usepackage{threeparttablex}%
%\usepackage[normalem]{ulem}%
\usepackage{makecell}%
%% Wrap long output lines
\usepackage{listings}%
\lstset{breaklines=true}%
\usepackage{enumitem}%
\setlist[description]{style=nextline}%
%% For nice info boxes
\usepackage{fontawesome5}%
\usepackage{awesomebox}%
\usepackage{siunitx}%
\newcolumntype{d}{S[table-format=3.2]}%
\renewcommand{\bibname}{References}%
%\usepackage[colorlinks]{hyperref}%

\usepackage{titling}%
%\setlength{\droptitle}{8cm}
\pretitle{\newpagecolor{dundeeblue}\afterpage{\restorepagecolor} \vfill \begin{flushleft}
\fontsize{68pt}{62pt} \color{white}\sffamily\bfseries\selectfont }
\posttitle{\end{flushleft}}

\preauthor{\vspace{2.5cm} \begin{flushleft} \fontsize{18pt}{14pt} \color{white}\sffamily\selectfont}
\postauthor{$\quad\bullet\quad$ehall001@dundee.ac.uk\end{flushleft}}

\predate{\begin{flushleft} \fontsize{18pt}{14pt} \color{white}\sffamily\selectfont}
\postdate{\\ \vspace{1cm}\includegraphics[width=10cm]{assets/images/rev_logo.pdf}\end{flushleft}}

%%% BEGIN SHORTCUTS
\DeclareMathOperator{\E}{\mathbf{E}}%
\DeclareMathOperator{\Var}{Var}%
\DeclareMathOperator{\Cov}{Cov}%
\DeclareMathOperator{\corr}{corr}%
\DeclareMathOperator{\sd}{sd}%
\newcommand{\se}{\mathsf{se}}%
%%% END SHORTCUTS

%% Change chapter to Topic
\makeatletter
\renewcommand{\@chapapp}{Topic}
\makeatother

\usepackage{tocbibind}



\usepackage{booktabs}
\usepackage{longtable}
\usepackage{array}
\usepackage{multirow}
\usepackage{wrapfig}
\usepackage{float}
\usepackage{colortbl}
\usepackage{pdflscape}
\usepackage{tabu}
\usepackage{threeparttable}
\usepackage{threeparttablex}
\usepackage[normalem]{ulem}
\usepackage{makecell}
\usepackage{xcolor}
\KOMAoption{captions}{tableheading}
\makeatletter
\@ifpackageloaded{tcolorbox}{}{\usepackage[skins,breakable]{tcolorbox}}
\@ifpackageloaded{fontawesome5}{}{\usepackage{fontawesome5}}
\definecolor{quarto-callout-color}{HTML}{909090}
\definecolor{quarto-callout-note-color}{HTML}{0758E5}
\definecolor{quarto-callout-important-color}{HTML}{CC1914}
\definecolor{quarto-callout-warning-color}{HTML}{EB9113}
\definecolor{quarto-callout-tip-color}{HTML}{00A047}
\definecolor{quarto-callout-caution-color}{HTML}{FC5300}
\definecolor{quarto-callout-color-frame}{HTML}{acacac}
\definecolor{quarto-callout-note-color-frame}{HTML}{4582ec}
\definecolor{quarto-callout-important-color-frame}{HTML}{d9534f}
\definecolor{quarto-callout-warning-color-frame}{HTML}{f0ad4e}
\definecolor{quarto-callout-tip-color-frame}{HTML}{02b875}
\definecolor{quarto-callout-caution-color-frame}{HTML}{fd7e14}
\makeatother
\makeatletter
\@ifpackageloaded{bookmark}{}{\usepackage{bookmark}}
\makeatother
\makeatletter
\@ifpackageloaded{caption}{}{\usepackage{caption}}
\AtBeginDocument{%
\ifdefined\contentsname
  \renewcommand*\contentsname{Table of contents}
\else
  \newcommand\contentsname{Table of contents}
\fi
\ifdefined\listfigurename
  \renewcommand*\listfigurename{List of Figures}
\else
  \newcommand\listfigurename{List of Figures}
\fi
\ifdefined\listtablename
  \renewcommand*\listtablename{List of Tables}
\else
  \newcommand\listtablename{List of Tables}
\fi
\ifdefined\figurename
  \renewcommand*\figurename{Figure}
\else
  \newcommand\figurename{Figure}
\fi
\ifdefined\tablename
  \renewcommand*\tablename{Table}
\else
  \newcommand\tablename{Table}
\fi
}
\@ifpackageloaded{float}{}{\usepackage{float}}
\floatstyle{ruled}
\@ifundefined{c@chapter}{\newfloat{codelisting}{h}{lop}}{\newfloat{codelisting}{h}{lop}[chapter]}
\floatname{codelisting}{Listing}
\newcommand*\listoflistings{\listof{codelisting}{List of Listings}}
\makeatother
\makeatletter
\makeatother
\makeatletter
\@ifpackageloaded{caption}{}{\usepackage{caption}}
\@ifpackageloaded{subcaption}{}{\usepackage{subcaption}}
\makeatother
\usepackage{bookmark}
\IfFileExists{xurl.sty}{\usepackage{xurl}}{} % add URL line breaks if available
\urlstyle{same}
\hypersetup{
  pdftitle={Problem Solving},
  pdfauthor={Dr Eric Hall},
  colorlinks=true,
  linkcolor={blue},
  filecolor={Maroon},
  citecolor={Blue},
  urlcolor={Blue},
  pdfcreator={LaTeX via pandoc}}


\title{Problem Solving}
\author{Dr Eric Hall}
\date{2025-08-19}
\begin{document}
\maketitle

\renewcommand*\contentsname{Table of contents}
{
\hypersetup{linkcolor=}
\setcounter{tocdepth}{2}
\tableofcontents
}

\bookmarksetup{startatroot}

\chapter*{Introduction}\label{introduction}
\addcontentsline{toc}{chapter}{Introduction}

\markboth{Introduction}{Introduction}

Welcome to PH11002 Problem Solving and Professional Skills at the
University of Dundee.

These notes are available at
\href{https://dundeemath.github.io/PH11002/}{dundeemath.github.io/PH11002/}
and also as a PDF (visit the page and click on the PDF icon to
download).

\section*{Licence}\label{licence}
\addcontentsline{toc}{section}{Licence}

\markright{Licence}

\pandocbounded{\includegraphics[keepaspectratio]{index_files/mediabag/88x31.png}}

This work is licensed under a
\href{http://creativecommons.org/licenses/by-nc/4.0/}{Creative Commons
Attribution-NonCommercial 4.0 International License}.

\part{Problem Solving}

\chapter{Purpose}\label{sec-purpose}

\begin{tcolorbox}[enhanced jigsaw, breakable, colframe=quarto-callout-tip-color-frame, leftrule=.75mm, arc=.35mm, toptitle=1mm, rightrule=.15mm, left=2mm, colbacktitle=quarto-callout-tip-color!10!white, bottomtitle=1mm, titlerule=0mm, bottomrule=.15mm, toprule=.15mm, coltitle=black, opacityback=0, title=\textcolor{quarto-callout-tip-color}{\faLightbulb}\hspace{0.5em}{What are we here for?}, opacitybacktitle=0.6, colback=white]

``{[}T{]}he mathematician's main reason for existence is to solve
problems {[}\ldots{]} therefore, what mathematics \emph{really} consists
of is problems and solutions'' {[}Halmos
(\citeproc{ref-Halmos:1980ps}{1980}); original emphasis{]}.

\end{tcolorbox}

Throughout your mathematics and scientific journey, you have likely
faced a variety of problems that provided essential context to guide you
toward the ``appropriate'' solution methods. Typically, you encounter a
specific topic, technique, or method, followed by a series of related
exercises that reinforce that concept. While this approach can be
beneficial for solidifying your understanding, it can also be somewhat
disconnected from real-world challenges. In reality, problems often
integrate multiple concepts and span various areas of mathematics,
requiring a more holistic application of your knowledge.

In this module, we want to develop skills that help you to solve
problems more generally. This will be done by engaging with tasks for
which the solution method is not known in advance. To get good at
attacking such open-ended tasks, you will also learn to reflect on your
experiences and problem solving processes. This is because
problem-solving is not simply a product of \emph{what you know} (i.e.,
your mathematical resources), it is also a function of your perceptions
of that knowledge that you derive from your \emph{experiences} with
mathematics (\citeproc{ref-Schoenfeld:1985ps}{Schoenfeld 1985}).

In this module, our focus is on developing skills that enhance your
ability to solve problems in a broader context. We will achieve this by
working on tasks for which the solution methods are not known in
advance. To excel in tackling these open-ended challenges, you will also
engage in reflecting on your experiences and the processes you use for
problem-solving. This is essential because effective problem-solving
relies not only on your mathematical resources, i.e., \emph{what you
know}, but also on how you perceive that knowledge
(\citeproc{ref-Schoenfeld:1985ps}{Schoenfeld 1985}). Your perception of
mathematics is influenced by your experiences, and a key aspect of this
module will involve engaging in problem-solving tasks, both individually
and collaboratively, and reflecting on these experiences to enhance your
learning.

The purpose of this problem solving module is to:

\begin{itemize}
\tightlist
\item
  build your confidence in solving unseen problems,
\item
  provide prompts to support reflection that will deepen your perception
  of mathematical knowledge (and its interconnections),
\item
  give generic support scaffolding problem solving (vs scaffolding the
  problem) that will provide a foundation for your development as a
  mathematician and scientist.
\end{itemize}

\chapter{Problems}\label{sec-problems}

\begin{quote}
``Problem. A doubtful or difficult question; a matter of inquiry,
discussion, or thought; a question that exercises the mind.''
(\citeproc{ref-OED:1989ps}{\emph{Oxford English Dictionary} 1989})
\end{quote}

Problems often involve more complexity than straightforward exercises in
that the method of solution is not proscribed. We will differentiate
between two types of problems and present a general framework,
introduced by Pólya (\citeproc{ref-Polya:1945hu}{Polya 1945}), for
understanding problem solving. This framework is intended to serve as a
foundation for analysing your approach to problem solving.

\section{Two Types of Problems}\label{two-types-of-problems}

We distinguish between different types of problems based on the desired
goal.

\textbf{Problems to find.} The task is to produce or create an object
that satisfies specified conditions, which may include a number,
function, construction, example, counterexample, or algorithm. Common
prompts for these tasks include terms such as determine, compute,
construct, and classify. The success of these endeavors is evaluated
based on criteria such as correctness, completeness, and in some cases,
optimality or uniqueness.

\begin{tcolorbox}[enhanced jigsaw, breakable, colframe=quarto-callout-note-color-frame, leftrule=.75mm, arc=.35mm, toptitle=1mm, rightrule=.15mm, left=2mm, colbacktitle=quarto-callout-note-color!10!white, bottomtitle=1mm, titlerule=0mm, bottomrule=.15mm, toprule=.15mm, coltitle=black, opacityback=0, title=\textcolor{quarto-callout-note-color}{\faInfo}\hspace{0.5em}{Example of a problem to find}, opacitybacktitle=0.6, colback=white]

Find all integers \(n\) such that \(n(n+1)\) is a perfect square.

\end{tcolorbox}

\textbf{Problems to prove.} The task is to justify a claim beyond
reasonable doubt. Typical prompts include: show, prove, disprove,
establish, and deduce. To be successful, one must ensure validity,
clarity, and appropriate use of definitions and prior results.

\begin{tcolorbox}[enhanced jigsaw, breakable, colframe=quarto-callout-note-color-frame, leftrule=.75mm, arc=.35mm, toptitle=1mm, rightrule=.15mm, left=2mm, colbacktitle=quarto-callout-note-color!10!white, bottomtitle=1mm, titlerule=0mm, bottomrule=.15mm, toprule=.15mm, coltitle=black, opacityback=0, title=\textcolor{quarto-callout-note-color}{\faInfo}\hspace{0.5em}{Example of a problem to prove}, opacitybacktitle=0.6, colback=white]

Prove there are infinitely many primes congruent to \(3 \mod 4\).

\end{tcolorbox}

\begin{tcolorbox}[enhanced jigsaw, breakable, colframe=quarto-callout-warning-color-frame, leftrule=.75mm, arc=.35mm, toptitle=1mm, rightrule=.15mm, left=2mm, colbacktitle=quarto-callout-warning-color!10!white, bottomtitle=1mm, titlerule=0mm, bottomrule=.15mm, toprule=.15mm, coltitle=black, opacityback=0, title=\textcolor{quarto-callout-warning-color}{\faExclamationTriangle}\hspace{0.5em}{Many tasks mix both!}, opacitybacktitle=0.6, colback=white]

Find all objects with property \(P\) and prove your list is complete.

\end{tcolorbox}

In this module, we will focus on \emph{problems to find}.

\section{Problem solving framework}\label{problem-solving-framework}

The problem solving framework, developed in
(\citeproc{ref-Polya:1945hu}{Polya 1945}), is a four-phase cycle for
tackling open-ended problems. The phases of the problem solving
framework are depicted in Figure~\ref{fig-phases}; the name of each
phase is listed in bold text, with key terms in normal text. Knowing
which phase of the framework you are in may help you choose the best
prompt to move forward (see the table below in
Section~\ref{sec-phases}).

\begin{figure}

\centering{

\includegraphics[width=2.75in,height=4.87in]{02-problems_files/figure-latex/mermaid-figure-1.png}

}

\caption{\label{fig-phases}Four phases of problem solving.}

\end{figure}%

The phases are roughly as follows. First, \textbf{understand the
problem}: identify givens, unknowns, and conditions; restate it in your
own words; sketch, tabulate, or probe small cases. Next, \textbf{devise
a plan} by choosing a route. Possible routes are to work backward, look
for patterns, simplify or specialise, use symmetry or invariants,
introduce an auxiliary object, estimate or bound, change representation,
or reduce to a known problem (we will investigate these problem solving
\emph{heuristics} later in Chapter~\ref{sec-heuristics}). Then
\textbf{carry out \emph{your} plan}: execute cleanly, justify each step,
check subgoals, and pivot if a step stalls. Finally, \textbf{look back}:
verify the result, test edge cases, assess efficiency and clarity, and
capture the key idea. The final phase is \emph{essential}. Use the
framework to reflect on what you learned so your future problem solving
gets faster and more reliable.

\section{Phases of problem solving}\label{sec-phases}

Use Table Table~\ref{tbl-phases} as a working checklist and a reflection
guide, not a rigid recipe. As you tackle an open-ended problem, you
might find that you are ``stuck''. First, don't panic! Decide which
phase of the problem solving framework you are in and scan the prompts
in that row. Answering the prompt may lead to a concrete next action.
Reflect on this new action for a few minutes; if it stalls, return to
the table, pick a different prompt, or take a break. Over time, the
prompts will become more familiar and the process of solving an
open-ended problme will be less daunting.

\begin{landscape}

\begin{longtable}[t]{p{0.18\linewidth}p{0.22\linewidth}p{0.30\linewidth}p{0.30\linewidth}}

\caption{\label{tbl-phases}Phases of problem solving, adapted from
(\citeproc{ref-Polya:1945hu}{Polya 1945, xvi--xvii}).}

\tabularnewline

\toprule
Phase & Purpose & Prompts to ask yourself & Useful tactics\\
\midrule
Understanding the problem & For a problem to find, understand the problem by making the unknown, data, and conditions precise. & What is unknown? What is given or what are the data? What conditions/constraints apply? Can I restate the task in my own words? What do small or extreme cases look like? What diagram/notation will help? & Define symbols. Draw a figure. List constraints. Test tiny cases. Identify edge cases. Rephrase the question. Separate various parts of the condition.\\
Devising a plan & Find the connection between the data and the unknown that provides a path towards a solution. & Have I seen the problem before? Have I seen the same problem in a slightly different form? Do I know a related problem? Do I know a theorem that might be useful? Can I work backward from the goal? Can I simplify/specialise first? What pattern, invariant, or symmetry might apply? Can I introduce an auxiliary element or change representation? Did I use all the data in devising my plan? Did I account for all the conditions? & Heuristics such as analogy; special/edge cases; generalisation; using invariants and symmetry; bounding/estimating; pigeonhole; substitution; set up equations or a new diagram.\\
Carry out your plan & Carry out plan, checking each step! & Would I be able to clearly explain that the step is correct? Can I prove that it is correct? Does each step follow from assumptions or known results? What subgoal can I verify now? If a step fails, which alternative route will I try next? & Justify steps. Prove lemmas. Compute carefully. Frequently take stock (checkpoint). Pivot quickly if a line of attack stalls. Don't panic.\\
Look back & Validate the result and reflect on learning. & Can I check the result? Can you explain your arguments? Does the result meet all conditions? Any counterexamples? Is it complete/optimal? Can I shorten or generalise the solution? Can I derive the result differently? What key idea made it work, and where else could it apply? & Record the central insight. Phases of the framework. Prompts answered. Heuristics used and how.\\
\bottomrule

\end{longtable}

\end{landscape}

\chapter{Heuristics}\label{sec-heuristics}

\section{What is a hueristic?}\label{what-is-a-hueristic}

heuristic as broad range of general problem-solving techniques.

heuristics: general problem solving strategies control: how one selects
and deploys the resources at one's disposal; the global decisions you
make regarding the selection and implementation of resources and
strategies

belief systems: perception of mathematical knowledge

\begin{quote}
``explicit mention of the heuristic techniques served to bring those
skills to the students' conscious attention and to help them codify and
reorganize their existing knowledge in such a way that those skills
could now be accessed and used more readily''
(\citeproc{ref-Schoenfeld:1985ps}{Schoenfeld 1985})
\end{quote}

\section{Dictionary of heuristics}\label{dictionary-of-heuristics}

Each of these heuristics is not precise enough to allow for unambiguous
interpretation and subsequent application to a particular problem!
Consider each as a \emph{label} for a closely related family of devices.
A challenge is to think about how the heuristic might be developed into
a specific strategy.

Decomposing heuristics into a targeted or specific strategies is the
first step in learning how use them.

\begin{tcolorbox}[enhanced jigsaw, breakable, colframe=quarto-callout-warning-color-frame, leftrule=.75mm, arc=.35mm, toptitle=1mm, rightrule=.15mm, left=2mm, colbacktitle=quarto-callout-warning-color!10!white, bottomtitle=1mm, titlerule=0mm, bottomrule=.15mm, toprule=.15mm, coltitle=black, opacityback=0, title=\textcolor{quarto-callout-warning-color}{\faExclamationTriangle}\hspace{0.5em}{Frameworks are not substitution for knowledge!}, opacitybacktitle=0.6, colback=white]

Heuristic strategies are not a subsitute for domain knowledge. ``Despite
the fact that their application cuts across various mathematical
domains, the successful implementation of heuristic strategies in any
particular domain often depends heavily on the possession of specific
subject matter knowledge.'' (S) Heuristics will not replace shaky
mastery of a subject!

\end{tcolorbox}

\begin{enumerate}
\def\labelenumi{(\Alph{enumi})}
\setcounter{enumi}{15}
\tightlist
\item
  = from Polya specifically {[}Polya:1945ps{]}
\item
  = from
\item
  = a modern/computational hueristic from {[}MichalewiczFogel:2004ps{]}
\end{enumerate}

\#\#\#~Variation of the problem (P)

\begin{itemize}
\tightlist
\item
  Decomposing and recombining
\item
  Establishing and using subgoals
\item
  Generalisation
\item
  Specialisation
\end{itemize}

``To better understand an unfamiliar problem, you may wish to exemplify
the problem by considering various special cases. This may suggest the
direction of, or perhaps the plausibility of, a solution.\textbar{} (S)

\begin{itemize}
\tightlist
\item
  Analogy
\end{itemize}

\subsection{Auxiliaries (P)}\label{auxiliaries-p}

\begin{itemize}
\tightlist
\item
  Auxiliary elements
\item
  Auxiliary problem
\end{itemize}

\subsection{Notation (P)}\label{notation-p}

\subsection{Figures (P)}\label{figures-p}

\subsection{Examine your guess (P)}\label{examine-your-guess-p}

\subsection{Working backwards (P)}\label{working-backwards-p}

\subsection{Setting up equations {[}as translation{]}
(P)}\label{setting-up-equations-as-translation-p}

\subsection{Check the result (P)}\label{check-the-result-p}

\subsection{Reductio ad absurdum and indirect proof
(P)}\label{reductio-ad-absurdum-and-indirect-proof-p}

\subsection{Type checking and dimensional analysis
(P)}\label{type-checking-and-dimensional-analysis-p}

\subsection{Approximation (modelling)}\label{approximation-modelling}

\subsection{Estimation}\label{estimation}

\subsection{Exhaustive search (M)}\label{exhaustive-search-m}

\subsection{Greedy algorithms (M)}\label{greedy-algorithms-m}

makes the best local choice at each step, hoping to find a global
optimum solution \#\#\# Neural Networks (M)

\chapter{Control}\label{sec-control}

Attitudes to problem solving. - Selecting and pursuing the right
approach and recovering from inapproriate choices

\begin{tcolorbox}[enhanced jigsaw, breakable, colframe=quarto-callout-important-color-frame, leftrule=.75mm, arc=.35mm, toptitle=1mm, rightrule=.15mm, left=2mm, colbacktitle=quarto-callout-important-color!10!white, bottomtitle=1mm, titlerule=0mm, bottomrule=.15mm, toprule=.15mm, coltitle=black, opacityback=0, title=\textcolor{quarto-callout-important-color}{\faExclamation}\hspace{0.5em}{Important}, opacitybacktitle=0.6, colback=white]

If you find yourself `stuck' try answering these questions
{[}Schoenfeld:1985ps{]}: - What exactly are you doing? Can you describe
it precisely? - Why are you doing it? How does it fit into the solution?
- How does it help you? What will you do with the outcome when you
obtain it? :::

\chapter*{References}\label{references}
\addcontentsline{toc}{chapter}{References}

\markboth{References}{References}

\begingroup
\raggedright

\phantomsection\label{refs}
\begin{CSLReferences}{1}{0}
\bibitem[\citeproctext]{ref-Halmos:1980ps}
Halmos, P. R. 1980. {``The Heart of Mathematics.''} \emph{The American
Mathematical Monthly} 87 (7): 519--24.
\url{https://doi.org/10.2307/2321415}.

\bibitem[\citeproctext]{ref-OED:1989ps}
\emph{Oxford English Dictionary}. 1989. 2nd ed. Oxford: Oxford
University Press.

\bibitem[\citeproctext]{ref-Polya:1945hu}
Polya, George. 1945. \emph{{How to Solve It}}. Princeton University
Press.

\bibitem[\citeproctext]{ref-Schoenfeld:1985ps}
Schoenfeld, Alan H. 1985. \emph{Mathematical Problem Solving}. Academic
Press. \url{https://doi.org/10.1016/C2013-0-05012-8}.

\end{CSLReferences}

\endgroup

\part{Professional Skills}

\end{tcolorbox}




\end{document}
